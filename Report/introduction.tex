\section{Introduction}

\subsection{Résumé}

\indent Ce rapport vise à présenter l’état d’avancement de notre projet de recherche de mots cachés. Nous débuterons par la description des différentes fonctionnalités développées jusqu'à présent, telles que le chargement d’images, la détection de la grille et la résolution de mots cachés. Nous expliquerons également le fonctionnement de notre réseau de neurones, illustrant son apprentissage avec une fonction logique simple. Enfin, nous conclurons par une démonstration pratique des éléments réalisés, en mettant en avant l’interaction entre les différentes parties du programme et les résultats obtenus.

\indent \newline\newline

\subsection{Présentation du groupe} \subsubsection{Corentin} \indent Je m'appelle Corentin... \newline\newline

\subsubsection{Arnaud} \indent Je m'appelle Arnaud... \newline\newline

\subsubsection{Adrien} \indent Je m'appelle Adrien, j'ai 19 ans. Mes motivations pour l'informatique sont d'apprendre de nouvelles choses et d'améliorer mes compétences dans le plus de domaines liés à l'informatique possible. Je suis très intéressé par l'intelligence artificielle sous toutes ses formes, notamment la reconnaissance optique. C'est pourquoi je m'occuperai principalement du réseau de neurones. \newline\newline

\subsection{Aspects humains} \indent Nous avons décidé de travailler ensemble parce que nous nous connaissions avant le projet ; nous sommes tous les trois amis. Cette amitié est un réel atout dans la réalisation d'un projet de groupe, car il n'y a aucun risque de tension. Nous avons également choisi de ne travailler qu'à trois pour des raisons de facilitation de la répartition des tâches principalement.
