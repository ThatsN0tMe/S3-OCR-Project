\section{Conclusion}
\subsection{État actuel}
\indent À ce jour, nous pensons être en mesure de terminer le projet dans les délais impartis. Le léger retard accumulé sur certaines tâches pourra être rattrapé sans difficulté. À l’heure actuelle, aucun problème majeur de développement ou d’organisation n'a été rencontré.

\subsection{Étapes à venir}
\indent Dans les prochaines étapes, nous allons implémenter une rotation automatique des images. Cette automatisation facilitera le traitement d’image et améliorera la fluidité de l’expérience utilisateur. \newline 
Nous allons également adapter notre réseau de neurones, qui est pour l’instant uniquement capable de résoudre des opérations logiques. Il sera modifié pour pouvoir reconnaître de petites images de lettres de 28x28 pixels. Nous entraînerons ce réseau avec le jeu de données EMNIST (Extended MNIST), qui contient plusieurs images de lettres au format 28x28 pixels. \newline
Une autre étape consistera à créer une interface graphique plus agréable et complète. Nous ajouterons diverses options de paramétrage, notamment pour le traitement de l’image, comme les seuils de détection, le contraste, etc. \newline
Enfin, si le temps le permet, nous aimerions également améliorer notre solveur pour qu'il soit compatible avec des graphes reconnaissant les mots ayant des préfixes similaires. Cela rendra le solveur bien plus performant.
